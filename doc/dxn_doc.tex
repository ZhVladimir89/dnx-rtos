\documentclass[a4paper,11pt]{article}
\usepackage[utf8]{inputenc}
\usepackage[sc]{mathpazo}
\usepackage[T1]{fontenc}
\usepackage[colorlinks=true,urlcolor=blue,linkcolor=red,citecolor=black]{hyperref}
\usepackage[left=40mm,right=30mm,top=30mm,bottom=30mm]{geometry}
\usepackage{url}
\linespread{1.05}

\title{dnx/FreeRTOS System Documentation}
\author{Daniel Zorychta}
\date{\today}

\begin{document}
\maketitle
\tableofcontents

\section{Introduction}
The dnx/FreeRTOS is a general purpose operating system based on FreeRTOS kernel.
The dnx layer is modeled on well-known Unix architecture (everything is a file).
Destination  of  the  system  are  small  microcontrollers supported by FreeRTOS
kernel,  especially  32-bit.  System  is easy scalable to user's needs, user can
write own drivers,  virtual devices,  programs and so on.  The programs layer is
mostly compatibile with C standard.

\section{Project Build}

\section{Driver Development}

\section{Program Development}

\section{File System Development}

\section{Limitations}

\end{document}
